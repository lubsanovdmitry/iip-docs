{
	\addcontentsline{toc}{chapter}{СПИСОК ИСТОЧНИКОВ}  
	\chapter*{Список источников}
	\begin{enumerate}
		\item Адельсон-Вельский, Г. М. Машина Играет в шахматы / Г. М. Адельсон-Вельский, В. Л. Арлазаров, А. Р. Битман, М. В. Донской; отв. ред. А. Ф. Волков. --- М.:Наука, 1983. --- 208 с.
		\item Байнов, Е. В. Шатар --- бурятские шахматы. / Е. В. Байнов, М. Е. Байнова. --- Улан-Удэ: Детско-юношеская спортивная школа №8, 2015. --- 26 с.
		\item Корнилов, Е. Н. Программирование шахмат и других логических игр. --- СПб.:БХВ-Петербург, 2005. --- 272 с. 
		\item Browne, Cameron. Bitboard Methods for Games // ICGA Journal. --- 2014. --- Vol. 37. --- No. 2. --- pp. 65-84.
		\item Chess Programming Wiki : [Electronic Resource]. --- \\URL: \href{https://chessprogrammingwiki.org}{https://chessprogrammingwiki.org.}
		\item rang: A Minimal, Header only Modern c++ library for terminal goodies // GitHub --- The world’s leading software development platform. --- Access mode: https://github.com/agauniyal/rang
		(access date: 17.03.2023).
		\item Shatar // Wikipedia. The Free Encyclopedia. --- Access mode: \\ https://en.wikipedia.org/wiki/Shatar
		(access date: 17.03.2023).
	\end{enumerate}
	\clearpage
}
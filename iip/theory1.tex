{

	\chapter{Обзор и теория}
	\section{Используемые технологии}
	Создание движка будет вестись на языке С++. Это мощный и  популярный язык для разработки настольных приложений.
	
	Разработка будет вестись в среде CLion от компании JetBrains. Она также является популярной, имеет широкий набор возможностей. Кроме того, производитель предоставляет бесплатные лицензии учащимся. 
	
	Компилятором является GNU G++ 12.2, а системой сборки --- CMake 3.26. Они оба имеют открытый исходный код и бесплатны. Проект создан при помощи дополнения cmake-init.
	
	Для управления терминальным выводом используется библиотека rang[6].
	
	\section{Обзор игры}
	
	Как уже было сказано, шатар --- это разновидность шахмат, и большинство правил схожи с европейскими шахматами. Так что имеет смысл описать различия.
	На самом деле, степень отличия может варьироваться, ведь среди шатар тоже встречаются разные варианты. 
	
	Возьмём некую <<усреднённую>> версию[2][]:
	
	\begin{itemize}
		\item Отсутствует рокировка;
		\item Пешка не может совершать двойное перемещение, кроме случаев начала игры. Таким образом, отсутствует взятие на проходе;
		\item В начале игры обязательно совершаются ходы d2d4 и d7d5 (можно считать это начальным состоянием игры);
		\item 
	\end{itemize}
	
	\section{Теория шахматных движков}
	
	Шахматный движок --- программа, переносящая игру в шахматы в форму компьютерной программы. Движки также обычно содержат в себе некий алгоритм, способный играть в шахматы.
	
	Тогда шахматный движок как бы состоит из двух частей: шахматной доски и алгоритма игры (условно --- искусственного интеллекта, ИИ).
	
	\subsection{Шахматная доска}
	
	Существует множество способов представления шахматной доски[5] --- например, можно было бы использовать <<наивный>>: создадим массив размером 8х8, в каждой ячейке которого находится некая фигура. Ещё лучше будет создать доску размером 10х12, тогда удобнее будет обработать фигуры, которые могут выйти за пределы поля.
	
	Однако, уже давно существует более удобный (в первую очередь для компьютера) способ, использующий т.н. <<битовые доски>> (англ. \textit{bitboards}). Его достоинства в том, что требуется меньше памяти, а также увеличивается скорость (за счёт использования встроенных операций). Так, можно сгенерировать ходы для пешек просто произведя побитовый сдвиг.
	
	Назовём \textit{битовой доской} двоичное число длины 64, \textit{i}-й бит которого обозначает, находится ли в позиции \textit{i} какая-либо фигура.
	
	Для удобства отметим, что нумерация битов начинается со старшего (most significant) бита. Тогда получится такое соответствие квадратам доски:
	
	%\vspace{1em}
	
	
	\begin{table}[h]
		\centering
		\caption{Соответствие}
		\label{tab: accord}
		\begin{tabular}{|l|l|l|l|l|l|l|l|l|}
			\hline
			  & A & B  & C  & D  & E  & F  & G  & H  \\ \hline
			8 & 0 & 8  & 16 & 24 & 32 & 40 & 48 & 56 \\ \hline
			7 & 1 & 9  & 17 & 25 & 33 & 41 & 49 & 57 \\ \hline
			6 & 2 & 10 & 18 & 26 & 34 & 42 & 50 & 58 \\ \hline
			5 & 3 & 11 & 19 & 27 & 35 & 43 & 51 & 59 \\ \hline
			4 & 4 & 12 & 20 & 28 & 36 & 44 & 52 & 60 \\ \hline
			3 & 5 & 13 & 21 & 29 & 37 & 45 & 53 & 61 \\ \hline
			2 & 6 & 14 & 22 & 30 & 38 & 46 & 54 & 62 \\ \hline
			1 & 7 & 15 & 23 & 31 & 39 & 47 & 55 & 63 \\ \hline
		\end{tabular}
	\end{table}
	
	Теперь шахматную доску можно представить как набор таких битовых досок --- по одной на каждый тип фигуры, ещё две, показывающие, какой стороне принадлежит фигура, а также --- общая, т.е. показывающая наличие или отсутствие фигуры на доске в принципе. К ним понадобятся ещё несколько вспомогательных (инвертированных).
	
	Доска и счётчик ходов составляют \textit{позицию}.
	
	Нам потребуется следить за повторяющимися позициями. Можно, конечно, хранить каждую позицию по отдельности, но это достаточно затратно. Поэтому можно каждую позицию \textit{хешировать}, т.е. присвоить \textit{хеш}. Такой хеш называют Zobrist-хешом[5].
	
	Для того, чтобы совершать ходы, нужна некая структура, описывающая ход. Она содержит в себе собственно ход, то, какой фигурой сходили, съеденную фигуру (если таковая была), продвижение (если оно было).
	
	Требуется реализовать правила. Сделаем так: для каждой позиции сгенерируем все возможный легальные ходы. Теперь, можно легко проверить ход на правильность найдя его в списке возможных ходов.
	
	Каждая позиция может быть инициализирована при помощи FEN-строки (строки в формате нотации Форсайта-Эдвардса --- нотации, описывающей шахматную позицию).
	
	\subsection{Алгоритм игры}
	 
	
}
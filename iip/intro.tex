{
	\addcontentsline{toc}{chapter}{ВВЕДЕНИЕ}  
	\chapter*{Введение}
	Шахматы издавна были популярны во всём мире. Очень часто разные народы вносили свои собственные изменения в правила (собственно, современные шахматы --- сильно изменившаяся древнеиндийская игра чатуранга). 
	
	Одним из таких вариантов является \textbf{<<Шатар>>} --- монгольские шахматы. Само название --- видоизменённое персидское <<шатранг>>, а то, в свою очередь, происходит из вышеупомянутого <<чатуранга>>.
	
	\textbf{\textit{Актуальность}} обусловлена вновь возросшим интересом к этому национальному виду шахмат. Открываются кружки, проводятся соревнования, турниры --- например, в рамках фестивалей <<Алтаргана>> и <<Сурхарбан>>. Они были впервые включены в их программу относительно недавно --- в 2008 году.
	
	Мною, к сожалению, не были найдены другие программные реализации этой игры.
	
	2022 год  объявлен Годом культурного наследия народов России; также многие направления определены в <<Концепции сохранения и развития нематериального культурного наследия народов Российской Федерации...>>. Проводятся мероприятия по сохранению культур народов. Развивается и культура Бурятии; потому  также важны  усилия по популяризации народных игр, в том числе и <<шатар>>.
	
	\textbf{\textit{Объект:}} монгольский вариант шахмат шатар.
	
	\textbf{\textit{Предмет:}} игровой движок шахмат шатар.
	
	\textbf{\textit{Цель проекта:}} написать на языке программирования C++ программу шахмат шатар.
	
	\textbf{\textit{Задачи:}}
	
	\begin{enumerate}
<<<<<<< HEAD
		\item Изучить правила шатар;
		\item Изучить теорию компьютерных шахмат;
		\item Создать программу, реализующую шатар.
=======
		\item Изучить информацию о шатар.
		\item Изучить теорию шахматных движков
		\item Написать шахматный движок на языке C++
>>>>>>> 418697d (Change)
	\end{enumerate}
	\clearpage
}